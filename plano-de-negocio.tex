%% abtex2-modelo-relatorio-tecnico.tex, v-1.9.7 laurocesar
%% Copyright 2012-2018 by abnTeX2 group at http://www.abntex.net.br/ 
%%
%% This work may be distributed and/or modified under the
%% conditions of the LaTeX Project Public License, either version 1.3
%% of this license or (at your option) any later version.
%% The latest version of this license is in
%%   http://www.latex-project.org/lppl.txt
%% and version 1.3 or later is part of all distributions of LaTeX
%% version 2005/12/01 or later.
%%
%% This work has the LPPL maintenance status `maintained'.
%% 
%% The Current Maintainer of this work is the abnTeX2 team, led
%% by Lauro César Araujo. Further information are available on 
%% http://www.abntex.net.br/
%%
%% This work consists of the files abntex2-modelo-relatorio-tecnico.tex,
%% abntex2-modelo-include-comandos and abntex2-modelo-references.bib
%%

% ------------------------------------------------------------------------
% ------------------------------------------------------------------------
% abnTeX2: Modelo de Relatório Técnico/Acadêmico em conformidade com 
% ABNT NBR 10719:2015 Informação e documentação - Relatório técnico e/ou
% científico - Apresentação
% ------------------------------------------------------------------------ 
% ------------------------------------------------------------------------

\documentclass[
	% -- opções da classe memoir --
	12pt,				% tamanho da fonte
	openright,			% capítulos começam em pág ímpar (insere página vazia caso preciso)
	twoside,			% para impressão em recto e verso. Oposto a oneside
	a4paper,			% tamanho do papel. 
	% -- opções da classe abntex2 --
	%chapter=TITLE,		% títulos de capítulos convertidos em letras maiúsculas
	%section=TITLE,		% títulos de seções convertidos em letras maiúsculas
	%subsection=TITLE,	% títulos de subseções convertidos em letras maiúsculas
	%subsubsection=TITLE,% títulos de subsubseções convertidos em letras maiúsculas
	% -- opções do pacote babel --
	english,			% idioma adicional para hifenização
	french,				% idioma adicional para hifenização
	spanish,			% idioma adicional para hifenização
	brazil,				% o último idioma é o principal do documento
	]{abntex2}


% ---
% PACOTES
% ---

% ---
% Pacotes fundamentais 
% ---
\usepackage{lmodern}			% Usa a fonte Latin Modern
\usepackage[T1]{fontenc}		% Selecao de codigos de fonte.
\usepackage[utf8]{inputenc}		% Codificacao do documento (conversão automática dos acentos)
\usepackage{indentfirst}		% Indenta o primeiro parágrafo de cada seção.
\usepackage{color}				% Controle das cores
\usepackage{graphicx}			% Inclusão de gráficos
\usepackage{microtype} 			% para melhorias de justificação
\usepackage{multirow}			% para mesclas linhas em tabelas

\usepackage[brazil]{babel}
\usepackage{graphicx}
\usepackage{subfig}

% ---

% ---
% Pacotes adicionais, usados no anexo do modelo de folha de identificação
% ---
\usepackage{multicol}
\usepackage{multirow}
% ---
	
% ---
% Pacotes adicionais, usados apenas no âmbito do Modelo Canônico do abnteX2
% ---
\usepackage{lipsum}				% para geração de dummy text
% ---

% ---
% Pacotes de citações
% ---
\usepackage[brazilian,hyperpageref]{backref}	 % Paginas com as citações na bibl
\usepackage[alf]{abntex2cite}	% Citações padrão ABNT

% --- 
% CONFIGURAÇÕES DE PACOTES
% --- 

% ---
% Configurações do pacote backref
% Usado sem a opção hyperpageref de backref
\renewcommand{\backrefpagesname}{Citado na(s) página(s):~}
% Texto padrão antes do número das páginas
\renewcommand{\backref}{}
% Define os textos da citação
\renewcommand*{\backrefalt}[4]{
	\ifcase #1 %
		Nenhuma citação no texto.%
	\or
		Citado na página #2.%
	\else
		Citado #1 vezes nas páginas #2.%
	\fi}%
% ---

% ---
% Informações de dados para CAPA e FOLHA DE ROSTO
% ---
\titulo{RELATÓRIO DE ESTÁGIO SUPERVISIONADO}
\autor{Universidade Federal do Amazonas \\ Faculdade de Tecnologia}
\local{Brasil}
\data{2019}
\instituicao{%
  Universidade Federal do Amazonas
  \par
  Intituto de Computação}
\tipotrabalho{Relatório técnico}
% O preambulo deve conter o tipo do trabalho, o objetivo, 
% o nome da instituição e a área de concentração 
\preambulo{Plano de Negócio da empresa \textit{GMG Solutions} para o aplicativo \textit{DangerMap} a ser apresentado na disciplina de Sistemas Distribuidos no ano de 2019.}
% ---

% ---
% Configurações de aparência do PDF final

% alterando o aspecto da cor azul
\definecolor{blue}{RGB}{41,5,195}

% informações do PDF
\makeatletter
\hypersetup{
     	%pagebackref=true,
		pdftitle={\@title}, 
		pdfauthor={\@author},
    	pdfsubject={\imprimirpreambulo},
	    pdfcreator={LaTeX with abnTeX2},
		pdfkeywords={abnt}{latex}{abntex}{abntex2}{relatório técnico}, 
		colorlinks=true,       		% false: boxed links; true: colored links
    	linkcolor=blue,          	% color of internal links
    	citecolor=blue,        		% color of links to bibliography
    	filecolor=magenta,      		% color of file links
		urlcolor=blue,
		bookmarksdepth=4
}
\makeatother
% --- 

% --- 
% Espaçamentos entre linhas e parágrafos 
% --- 

% O tamanho do parágrafo é dado por:
\setlength{\parindent}{1.3cm}

% Controle do espaçamento entre um parágrafo e outro:
\setlength{\parskip}{0.2cm}  % tente também \onelineskip

% ---
% compila o indice
% ---
\makeindex
% ---

% ----
% Início do documento
% ----
\begin{document}

% Seleciona o idioma do documento (conforme pacotes do babel)
%\selectlanguage{english}
\selectlanguage{brazil}

% Retira espaço extra obsoleto entre as frases.
\frenchspacing 

% ----------------------------------------------------------
% ELEMENTOS PRÉ-TEXTUAIS
% ----------------------------------------------------------
% \pretextual

% ---
% Capa
% ---
\imprimircapa
% ---

% ---
% Folha de rosto
% (o * indica que haverá a ficha bibliográfica)
% ---
\imprimirfolhaderosto*
% ---

% resumo na língua vernácula (obrigatório)
\setlength{\absparsep}{18pt} % ajusta o espaçamento dos parágrafos do resumo
% ---


% inserir o sumario
% ---
\pdfbookmark[0]{\contentsname}{toc}
\tableofcontents*
\cleardoublepage
% ---


\chapter{Introdução}

No presente projeto foi realizado o estudo e desenvolvimento de um sistema web para gerenciamento de monografias e download de certificados para os cursos de Engenharia Engenharia Elétrica - Eletrotécinica. O sistema visa facilitar e agilizar o processo de organização e disponibilização das monografias e declaração de conclusão de curso.

Para a implementação do projeto, foi necessário verificar junto ao supervisor e administrador de curso os requisitos necessários ao sistema e realizar pequenas entregas para que pudesse ser gerado um produto útil e de fácil uso, uma vez que apenas a ideia geral estava definida no momento do início do estágio. Também foi necessário realizar a criação de um domínio próprio do curso, composto de servidor de arquivos e banco de dados e a instalação do framework escolhido neste servidor.

Inicialmente foi definido que o sistema seria composto por páginas informativas, controle de acesso de usuário, módulos de busca, inserção e remoção de monografias, inserção e remoção dos professores, geração de declarações de conclusão de curso e autenticação automática das declarações, divulgação de eventos e notícias relacionadas ao curso, disponibilização de formulários que interessam ao discente do curso e resumo descritivo dos laboratórios.

No próximo capítulo será mostrado o produto e as funcionalidades desenvolvidas durante o período do estágio.



\chapter{Desenvolvimento}


Esse projeto foi desenvolvido com base no framework Yii2, uma vez que este já era de conhecimento do aluno e o seu uso ofereceria um desenvolvimento rápido e eficiente do sistema. Nos próximos subcapítulos será mostrado as tecnologias utilizadas e os módulos desenvolvidos.

\section{Framework Yii2}

Framework de software é um conjunto de códigos genéricos, geralmente em forma de classes, implementado em alguma linguagem de programação, usados para resolver um problema de um domínio específico. Atua onde há funcionalidades em comum entre várias aplicações, utilizando o conceito chamado de reúso em engenharia de software, o que reduz bastante o esforço e o tempo necessário para desenvolver um trabalho comum, como, por exemplo, acesso a banco de dados e validação de dados [3,4].

Yii2 é um framework PHP baseado em componentização de alta performance destinado a desenvolvimento rápido de modernas aplicações web [5]. O nome Yii, pronunciado ji, significa “simples e evolucionário” na língua chinesa. Também pode ser pensado como um acrônimo do termo em inglês Yes It Is! Por ser um framework de desenvolvimento genérico, pode ser utilizado para o desenvolvimento de todos os tipos de aplicações web com o uso de PHP. Como é baseado em uma arquitetura componentizada e um sofisticado suporte a caching, é especialmente adequado para o desenvolvimento de aplicações de grande escala, como portais, fóruns, sistemas de gerenciamento, serviços web RESTful e outros. Dentre as suas principais características, podem ser citadas modelo de arquitetura Model-View-Controller, Query builders e ActiveRecord para transações com banco de dados de forma rápida e fácil, suporte a desenvolvimento de API RESTful, armazenamento de cache de várias camadas e arquitetura expansível por meio de extensões. Seu código é aberto e é desenvolvido por uma grande equipe e possui uma larga comunidade de profissionais constantemente contribuindo com desenvolvedores de aplicações, além de dispor de uma enorme quantidade de tutoriais e materiais de apoio na internet. Outro ponto importante é que a cada atualização, seu código fonte incorpora as últimas tendências e melhores práticas de desenvolvimento web.

\section{MySql}

Um sistema de gerenciamento de banco de dados é um programa de computador que fornece recursos capazes de manipular informações em um banco de dados de forma a facilitar e abstrair a interação do usuário com as informações. São usados em muitas aplicações onde é necessário realizar o armazenamento e a sua posterior recuperação, como, por exemplo, sistemas de gerenciamento empresarial ou programas de correio eletrônico. Englobam um conjunto de procedimentos, que o usuário percebe como uma única ação, que garantem a integridade das transações realizadas, identificados pela sigla ACID, oriunda dos termos atomicidade, consistência, isolamento e durabilidade. Geralmente essas ações são executadas por uma linguagem própria, sendo a SQL, sigla para Structured Query Language, a mais difundida atualmente.

MySql é um sistema de gerenciamento de banco de dados de uso livre baseado em linguagem sql que oferece muitas funcionalidades como triggers, views e connectors. Possui uma grande comunidade na internet, além vasta documentação e tutoriais que tornam a sua curva de aprendizado pequena e sua aplicação fácil em projetos, tanto por que é iniciante quanto profissional na área.

\section{Sistema de Gerenciamento de Monografias}

O sistema é composto por cinco módulos principais, que são: Professores, Notícias, Monografias, e Usuários. Cada um deles será descrito em detalhes ao longo deste relatório. O layout padrão é formado por um menu horizontal no topo da página, uma imagem do braço robótico do laboratório de controle com o texto “Engenharia Elétrica Eletrotécnica” e o corpo do sistema no restante da página, como é mostrado na figura 1.

O menu horizontal é composto por um link que retorna à página inicial do lado esquerdo, pelo botão Institucional que possui links para histórico, corpo docente, secretaria e laboratórios;, pelo botão Acadêmico que possui links para centro acadêmico e monografias;, pelo botão Coordenação que possui links para PPC, planejamento estratégico, disciplinas ofertadas; e contatos e por um botão de login/logout do lado direito. O corpo do sistema é o local onde o usuário executas as funções desejadas disponíveis nos módulos.

Há dois tipos de usuários possíveis, visitante e administrador. Cada qual tem uma visualização e acesso diferenciados de acordo com as suas permissões. A figura 2 mostra um diagrama de caso de uso com as ações permitidas a cada um deles.

No módulo Professores é possível cadastrar os professores que compõem o departamento. O usuário visitante pode visualizar a relação de professores cadastrados com as informações mais comuns, como foto, nome, descrição de sua carreira acadêmica, linha de pesquisa, atual formação e local de formação, e-mail e currículo lattes. O usuário administrador pode visualizar todos essas informações além de poder inserir, remover e editar os registros. A figura 3 mostra um exemplo do ponto de vista do perfil de administrador. Os dados deste módulo tem o objetivo principal de compor as informações que fazem parte da emissão da declaração e composição da banca de defesa de monografia.

No módulo Notícias é possível cadastrar as notícias gerenciadas pelo sistema. O usuário visitante não tem acesso na página principal às suas notícias, porém, não tem o poder de cadastrar novas notícias. Assim como no módulo anterior, o usuário administrador pode executar todas as ações do processo conhecido como CRUD, sigla para Inserir (Create), Remover (Remove), Atualizar (Update) e Apagar (Delete). Da mesma forma, os dados deste módulo tem o objetivo principal de compor as informações para a emissão da declaração de conclusão de curso.

No módulo Usuários são cadastrados os administradores de todos os cursos gerenciados pelo sistema. O perfil de visitante também não tem acesso a este módulo, assim como no módulo anterior. O perfil de administrador pode realizar todas as operações do CRUD e os seus dados também são usados para a geração automática da declaração de conclusão de curso, como o nome do administrador e e-mail impressos no final do documento. Vale ressaltar aqui que cada administrador pode apenas inserir as monografias referentes aos alunos do seu curso.

No módulo Monografias, o principal do sistema, é reunido todas as informações dos outros módulos importantes para o gerenciamento das monografias. Além das operações de CRUD típicas, é possível realizar o envio e download do arquivo da monografia no formato PDF, a alteração do token de autenticação da declaração e a visualização e download do arquivo da declaração de conclusão de curso.

O token de autenticação é um código gerado no momento do cadastro da monografia que tem a função de agilizar o processo de emissão do documento de declaração por parte do autor da monografia e da conferência a autenticidade deste documento por parte de um terceiro interessado, como, por exemplo, uma empresa na qual o aluno entregou este documento, sem a necessidade de intervenção direta e assinatura do coordenado do curso.

A figura 6 mostra a tela inicial deste módulo onde é possível selecionar as ações do CRUD. Os nomes das colunas, quando são clicados, também de a função de inverter e reorganizar a ordem que os dados são apresentados. Logo abaixo há campos de texto para a realização de filtros para facilitar e agilizar o processo de busca das monografias de interesse. Caso o número de monografias seja superior a vinte, é mostrado na parte inferior vários links para as páginas seguintes, visando facilitar e agilizar ainda mais o processo de busca.

A figura 7 mostra a tela de visualização. Nela é possível ver informações mais detalhadas sobre cada monografia. Os botões na parte superior são visíveis apenas pelos usuários do perfil administrador e os botões na parte inferior são visíveis por todos os usuários do sistema. O arquivo da monografia é armazenado no servidor de arquivos e fica disponível a todos os usuários interessados. A declaração de conclusão de curso é gerada em tempo de execução e fica disponível para download apenas no momento da sua requisição.


\chapter{Considerações Finais}

Devido à utilização do conceito de reúso, principalmente com a utilização do framework Yii2 foi possível a realização deste projeto no curto período destinado ao estágio obrigatório de conclusão de curso. Também devido à utilização do conceito de metodologia ágil, foi possível desenvolver um produto muito mais funcional, uma vez que existia apenas a ideia geral do sistema no início de sua implementação e os detalhes e particularidades de cada módulo foram sendo definidos e aperfeiçoados no decorrer do trabalho. Apesar de resolver bem o problema ao qual foi destinado, há ainda muito mais a ser feito neste sistema, podendo ser realizado por outros alunos, uma fez que as ferramentas utilizadas são de uso livre, de pequena curva de aprendizado e possuem muito material didático online disponível para consulta.

\end{document}
